\documentclass[twoside]{article}

\usepackage[english]{babel}
%\usepackage{a4}
\usepackage{amssymb,amsmath,bm}
\usepackage{graphicx,tabularx}
\usepackage{fancyhdr}
\usepackage{array}
\usepackage{float}
\usepackage{hyperref}
\usepackage{xspace}
\usepackage{rotating}
\usepackage{dcolumn}
\usepackage{geometry}
\usepackage{color}

\geometry{a4paper,left=20mm,right=20mm,top=20mm,bottom=20mm}

% \setlength{\topmargin}{10mm}
% \setlength{\topmargin}{-13mm}
% % \setlength{\oddsidemargin}{0.5cm}
% % \setlength{\evensidemargin}{0cm}
% \setlength{\oddsidemargin}{1cm}
% \setlength{\evensidemargin}{1cm}
% \setlength{\textwidth}{15cm}
\setlength{\textheight}{23.8cm}

\pagestyle{fancyplain}
\addtolength{\headwidth}{0.6cm}
\fancyhead{}%
\fancyhead[RE,LO]{\bf \textsc{GapIntegrals}}%
\fancyhead[LE,RO]{\thepage}
\cfoot{--- A.~Suter -- \today~ ---}
\rfoot{\includegraphics[width=2cm]{PSI-Logo_narrow.jpg}}

\DeclareMathAlphabet{\bi}{OML}{cmm}{b}{it}

\newcommand{\mean}[1]{\langle #1 \rangle}
\newcommand{\ie}{\emph{i.e.}\xspace}
\newcommand{\musrfithead}{MUSRFIT\xspace}
\newcommand{\musrfit}{\textsc{musrfit}\xspace}

\newcolumntype{d}[1]{D{.}{.}{#1}}
\newcolumntype{C}[1]{>{\centering\arraybackslash}p{#1}}

\begin{document}
% Header info --------------------------------------------------
\thispagestyle{empty}
\noindent
\begin{tabular}{@{\hspace{-0.2cm}}l@{\hspace{6cm}}r}
\noindent\includegraphics[width=3.4cm]{PSI-Logo_narrow.jpg} &
  {\Huge\sf Memorandum}
\end{tabular}
%
\vskip 1cm
%
\begin{tabular}{@{\hspace{-0.5cm}}ll@{\hspace{4cm}}ll}
Date:    & \today       &     & \\[3ex]
From:    & A. Suter     &     & \\
E-Mail:  & \verb?andreas.suter@psi.ch? &&
\end{tabular}
%
\vskip 0.3cm
\noindent\hrulefill
\vskip 1cm
%
\section*{Homogenous Disorder Model: GbG in Longitudinal Fields}%

Noakes and Kalvius \cite{noakes1997} derived a phenomenological model for
homogenous disorder: Gaussian-broadened Gaussian disorder (see also
Ref.\,\cite{yaouanc2011}). In both mentioned references only the zero field
case and the weak transverse field case are discussed. Here I briefly summarize
the longitudinal field (LF) case under the assumption that the applied field doesn't 
polarize the impurties, \ie the applied field is ``innocent''.

The Gauss-Kubo-Toyabe LF polarization function is

\begin{eqnarray}\label{eq:GKT_LF}
  P_{Z,{\rm GKT}}^{\rm LF} &=& 1 - 2 \frac{\sigma^2}{\omega_{\rm ext}^2}\left[ 1 - \cos(\omega_{\rm ext} t)\,\exp\left(-1/2 (\sigma t)^2\right) \right] + \label{eq:GKT_LF_1}\\
   & & + 2 \frac{\sigma^2}{\omega_{\rm ext}^3} \int_0^t \sin(\omega_{\rm ext} \tau)\,\exp\left(-1/2 (\omega_{\rm ext} \tau)^2\right) d\tau. \label{eq:GKT_LF_2}
\end{eqnarray}

\noindent The Gaussian disorder is assumed to have the funtional form 

\begin{equation}\label{eq:GaussianDisorder}
 \varrho = \frac{1}{\sqrt{2\pi}}\,\frac{1}{\sigma_1} \exp\left( -\frac{1}{2} \, \left[ \frac{\sigma - \sigma_0}{\sigma_1} \right]^2 \right).
\end{equation}

\noindent In Ref.\cite{yaouanc2011} a slightly different notation is used: $\sigma \to \Delta_{\rm G}$,  $\sigma_0 \to \Delta_{0}$, and 
$\sigma_1 \to \Delta_{\rm GbG}$.

\noindent The GbG LF polarization function is given by

\begin{equation}
 P_{Z,{\rm GbG}}^{\rm LF} = \int_0^\infty d\sigma \left\{ \varrho \cdot P_{Z,{\rm GKT}}^{\rm LF} \right\}.
\end{equation}

\noindent Assuming that $\sigma_0 \gg \sigma_1$ this can be approximated by

\begin{equation}
 P_{Z,{\rm GbG}}^{\rm LF} \simeq \int_{-\infty}^\infty d\sigma \left\{ \varrho \cdot P_{Z,{\rm GKT}}^{\rm LF} \right\}.
\end{equation}

\noindent Integrating 

\begin{equation*}
 P_{Z,{\rm GbG}}^{\rm LF, (1)} = \int_{-\infty}^\infty d\sigma \left\{ \varrho \cdot P_{Z,{\rm GKT}}^{\rm LF, (1)} \right\},  
\end{equation*}

\noindent where $P_{Z,{\rm GKT}}^{\rm LF, (1)}$ is given by Eq.(\ref{eq:GKT_LF_1}), leads to 

\begin{equation}\label{eq:GbG_LF_1}
 P_{Z,{\rm GbG}}^{\rm LF, (1)} = 1 - 2 \frac{\sigma_0^2+\sigma_1^2}{\omega_{\rm ext}^2} + 
      2 \frac{\sigma_0^2 + \sigma_1^2 (1 + \sigma_1^2 t^2)}{\omega_{\rm ext}^2 (1 + \sigma_1^2 t^2)^{5/2}}\, \cos(\omega_{\rm ext} t)\,
      \exp\left[-\frac{1}{2} \frac{\sigma_0^2 t^2}{1+\sigma_1^2 t^2}\right],
\end{equation}

\noindent and Eq.(\ref{eq:GKT_LF_2}) leads to the non-analytic integral

\begin{eqnarray}
  P_{Z,{\rm GbG}}^{\rm LF, (2)} &=& \int_{-\infty}^\infty d\sigma \left\{ \varrho \cdot P_{Z,{\rm GKT}}^{\rm LF, (2)} \right\} \nonumber \\
    &=& \int_0^t d\tau \left\{ \frac{\sigma_0^4 + 3 \sigma_1^4 (1 + \sigma_1^2 \tau^2)^2 + 6 \sigma_0^2 \sigma_1^2 (1+\sigma_1^2 \tau^2)}{\omega_{\rm ext}^3 (1+\sigma_1^2 \tau^2)^{9/2}} 
   \sin(\omega_{\rm ext} \tau)\, \exp\left[-\frac{1}{2} \frac{\sigma_0^2 t^2}{1+\sigma_1^2 t^2}\right] \right\}. \label{eq:GbG_LF_2}
\end{eqnarray}

\noindent The full GbG LF polarization function is hence

\begin{equation}
   P_{Z,{\rm GbG}}^{\rm LF} = P_{Z,{\rm GbG}}^{\rm LF, (1)} + P_{Z,{\rm GbG}}^{\rm LF, (2)}
\end{equation}


\subsection*{The GbG LF Polarization Function as a User Function in \musrfithead}

Eqs.(\ref{eq:GbG_LF_1})\&(\ref{eq:GbG_LF_2}) are implemented in \musrfit as user function. The current implementation is far from being efficient but stable.
The typical call from within the msr-file would be

\begin{verbatim}
###############################################################
FITPARAMETER
#      Nr. Name        Value     Step      Pos_Error  Boundaries
        1 PlusOne     1         0           none
        2 MinusOne    -1        0           none
        3 Alpha       0.78699   -0.00036    0.00036     0       none
        4 Asy         0.06682   0.00027     none        0       0.33
        5 Sig0        0.3046    -0.0087     0.0093      0       100
        6 Rb          1.0000    0.0027      none        0       1
        7 Field0      0         0           none
        8 Field1      20.03     0           none
        9 Field2      99.32     0           none

###############################################################
THEORY
asymmetry   fun1
userFcn  libGbGLF PGbGLF map2 5 fun2   (field sigma0 Rb)

###############################################################
FUNCTIONS
fun1 = map1 * par4
fun2 = par5 * par6
\end{verbatim}

\noindent where \texttt{PGbGLF} takes 3 arguments: 

\begin{enumerate}
 \item field in Gauss
 \item $\sigma_0$ in ($1/\mu s$)
 \item $R_b = \sigma_1 / \sigma_0$
\end{enumerate}

\noindent \textbf{Be aware that we explicitly assumed $\sigma_1 \ll \sigma_0$, \ie $R_b < 1$.}

\bibliographystyle{plain}
\begin{thebibliography}{1}

\bibitem{noakes1997} D.~R.~Noakes, G.~M.~Kalvius, Phys.~Rev.~B, \textbf{56}, 2352
(1997).
\bibitem{yaouanc2011} A.~Yaouanc, P.~Dalmas~de~R\'{e}otier, ``Muon Spin
Rotation, Relaxation, and Resonance'', Oxford University Press (2011). 

\end{thebibliography}


\end{document}
