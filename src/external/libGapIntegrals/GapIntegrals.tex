\documentclass[twoside]{article}

\usepackage[english]{babel}
%\usepackage{a4}
\usepackage{amssymb,amsmath,bm}
\usepackage{graphicx,tabularx}
\usepackage{fancyhdr}
\usepackage{array}
\usepackage{float}
\usepackage{hyperref}
\usepackage{xspace}
\usepackage{rotating}
\usepackage{dcolumn}
\usepackage{geometry}
\usepackage{color}

\geometry{a4paper,left=20mm,right=20mm,top=20mm,bottom=20mm}

% \setlength{\topmargin}{10mm}
% \setlength{\topmargin}{-13mm}
% % \setlength{\oddsidemargin}{0.5cm}
% % \setlength{\evensidemargin}{0cm}
% \setlength{\oddsidemargin}{1cm}
% \setlength{\evensidemargin}{1cm}
% \setlength{\textwidth}{15cm}
\setlength{\textheight}{23.8cm}

\pagestyle{fancyplain}
\addtolength{\headwidth}{0.6cm}
\fancyhead{}%
\fancyhead[RE,LO]{\bf \textsc{GapIntegrals}}%
\fancyhead[LE,RO]{\thepage}
\cfoot{--- B.M.~Wojek / A.~Suter -- \today~ ---}
\rfoot{\includegraphics[width=2cm]{PSI-Logo_narrow.jpg}}

\DeclareMathAlphabet{\bi}{OML}{cmm}{b}{it}

\newcommand{\mean}[1]{\langle #1 \rangle}
\newcommand{\lem}{LE-$\mu$SR\xspace}
\newcommand{\lemhead}{LE-$\bm{\mu}$SR\xspace}
\newcommand{\musr}{$\mu$SR\xspace}
\newcommand{\musrhead}{$\bm{\mu}$SR\xspace}
\newcommand{\trimsp}{\textsc{trim.sp}\xspace}
\newcommand{\musrfithead}{MUSRFIT\xspace}
\newcommand{\musrfit}{\textsc{musrfit}\xspace}
\newcommand{\gapint}{\textsc{GapIntegrals}\xspace}
\newcommand{\YBCO}{YBa$_{2}$Cu$_{3}$O$_{7-\delta}$\xspace}
\newcommand{\YBCOhead}{YBa$_{\bm{2}}$Cu$_{\bm{3}}$O$_{\bm{7-\delta}}$\xspace}

\newcolumntype{d}[1]{D{.}{.}{#1}}
\newcolumntype{C}[1]{>{\centering\arraybackslash}p{#1}}

\begin{document}
% Header info --------------------------------------------------
\thispagestyle{empty}
\noindent
\begin{tabular}{@{\hspace{-0.2cm}}l@{\hspace{6cm}}r}
\noindent\includegraphics[width=3.4cm]{PSI-Logo_narrow.jpg} &
  {\Huge\sf Memorandum}
\end{tabular}
%
\vskip 1cm
%
\begin{tabular}{@{\hspace{-0.5cm}}ll@{\hspace{4cm}}ll}
Date:    & \today       &     & \\[3ex]
From:    & B.M.~Wojek / Modified by A. Suter & \\
E-Mail:  & \verb?andreas.suter@psi.ch? &&
\end{tabular}
%
\vskip 0.3cm
\noindent\hrulefill
\vskip 1cm
%
\section*{\musrfithead plug-in for the calculation of the temperature dependence of $\bm{1/\lambda^2}$ for various gap symmetries}%

This memo is intended to give a short summary of the background on which the \gapint plug-in for \musrfit \cite{musrfit} has been developed. The aim of this implementation is the efficient calculation of integrals of the form
\begin{equation}\label{eq:int_phi}
 I(T) = 1 + \frac{1}{\pi}\int_0^{2\pi}\int_{\Delta(\varphi,T)}^{\infty}\left(\frac{\partial f}{\partial E}\right) \frac{E}{\sqrt{E^2-\Delta^2(\varphi,T)}}\mathrm{d}E\mathrm{d}\varphi\,,
\end{equation}
where $f = (1+\exp(E/k_{\mathrm B}T))^{-1}$, like they appear e.g. in the theoretical temperature dependence of $1/\lambda^2$~\cite{Manzano}.
For gap symmetries which involve not only a $E$- and $\varphi$-dependence but also a $\theta$-dependence, see the special section towards the end of the memo.
In order not to do too many unnecessary function calls during the final numerical evaluation we simplify the integral (\ref{eq:int_phi}) as far as possible analytically. The derivative of $f$ is given by
\begin{equation}\label{derivative}
\frac{\partial f}{\partial E} = -\frac{1}{k_{\mathrm B}T}\frac{\exp(E/k_{\mathrm B}T)}{\left(1+\exp(E/k_{\mathrm B}T)\right)^2} = -\frac{1}{4k_{\mathrm B}T} \frac{1}{\cosh^2\left(E/2k_{\mathrm B}T\right)}.
\end{equation}
Using (\ref{derivative}) and doing the substitution $E'^2 = E^2-\Delta^2(\varphi,T)$, equation (\ref{eq:int_phi}) can be written as
\begin{equation}\label{eq:bmw_2d}
\begin{split}
I(T) & = 1 - \frac{1}{4\pi k_{\mathrm B}T}\int_0^{2\pi}\int_{\Delta(\varphi,T)}^{\infty}\frac{1}{\cosh^2\left(E/2k_{\mathrm B}T\right)}\frac{E}{\sqrt{E^2-\Delta^2(\varphi,T)}}\mathrm{d}E\mathrm{d}\varphi \\
& = 1 - \frac{1}{4\pi k_{\mathrm B}T}\int_0^{2\pi}\int_{0}^{\infty}\frac{1}{\cosh^2\left(\sqrt{E'^2+\Delta^2(\varphi,T)}/2k_{\mathrm B}T\right)}\mathrm{d}E'\mathrm{d}\varphi\,.
\end{split}
\end{equation}
Since a numerical integration should be performed and the function to be integrated is exponentially approaching zero for $E'\rightarrow\infty$ the infinite $E'$ integration limit can be replaced by a cutoff energy $E_{\mathrm c}$ which has to be chosen big enough:
\begin{equation}
I(T) \simeq \tilde{I}(T) \equiv 1 - \frac{1}{4\pi k_{\mathrm B}T}\int_0^{2\pi}\int_{0}^{E_{\mathrm c}}\frac{1}{\cosh^2\left(\sqrt{E'^2+\Delta^2(\varphi,T)}/2k_{\mathrm B}T\right)}\mathrm{d}E'\mathrm{d}\varphi\,.
\end{equation}
In the case that $\Delta^2(\varphi,T)$ is periodic in $\varphi$ with a period of $\pi/2$ (valid for all gap symmetries implemented at the moment), it is enough to limit the $\varphi$-integration to one period and to multiply the result by $4$:
\begin{equation}
\tilde{I}(T) = 1 - \frac{1}{\pi k_{\mathrm B}T}\int_0^{\pi/2}\int_{0}^{E_{\mathrm c}}\frac{1}{\cosh^2\left(\sqrt{E'^2+\Delta^2(\varphi,T)}/2k_{\mathrm B}T\right)}\mathrm{d}E'\mathrm{d}\varphi\,.
\end{equation}
For the numerical integration we use algorithms of the \textsc{Cuba} library \cite{cuba} which require to have a Riemann integral over the unit square. Therefore, we have to scale the integrand by the upper limits of the integrations. Note that $E_{\mathrm c}$ and $\pi/2$ (or in general the upper limit of the $\varphi$ integration) are now treated as dimensionless scaling factors.
\begin{equation}
\tilde{I}(T) =  1 - \frac{E_{\mathrm c}}{2k_{\mathrm B}T}\int_0^{1_{\varphi}}\int_{0}^{1_{E}}\frac{1}{\cosh^2\left(\sqrt{(E_{\mathrm c}E)^2+\Delta^2\left(\frac{\pi}{2}\varphi,T\right)}/2k_{\mathrm B}T\right)}\mathrm{d}E\mathrm{d}\varphi
\end{equation}

\subsection*{Implemented gap functions and function calls from MUSRFIT}
Currently the calculation of $\tilde{I}(T)$ is implemented for various gap functions.
The temperature dependence of the gap functions is either given by Eq.(\ref{eq:gapT_Prozorov}) \cite{Prozorov}, or by Eq.(\ref{eq:gapT_Manzano}) \cite{Manzano}.

\vspace{2mm}

\noindent \color{red}\textbf{A few words of warning:~}\color{black} The temperature dependence of the gap function is typically derived from within the BCS framework, 
and strongly links $T_c$ and $\Delta_0$ (e.g.\xspace $\Delta_0 = 1.76\, k_{\rm B} T_c$ for an s-wave superconductor). In a self-consistent 
description this would mean that $\Delta_0$ of $\Delta(\varphi)$ is locked to $T_c$ as well. In the implementation provided, this limitation 
is lifted, and therefore the \emph{user} should judge and question the result if the ratio $\Delta_0/(k_{\rm B}T_c)$ is strongly deviating from
BCS values! 

\begin{equation}\label{eq:gapT_Prozorov}
 \Delta(\varphi,T) \simeq \Delta(\varphi)\,\tanh\left[c_0\,\sqrt{a_{\rm G} \left(\frac{T_{\rm c}}{T}-1\right)}\right]
\end{equation}
\noindent with $\Delta(\varphi)$ as given below, and $c_0$ and $a_{\rm G}$ depends on the pairing state:

\begin{description}
 \item [\textit{s}-wave:] $a_{\rm G}=1$ \qquad with $c_0 =  \frac{\displaystyle\pi k_{\rm B} T_{\rm c}}{\displaystyle\Delta_0} = \pi/1.76 = 1.785$
 \item [\textit{d}-wave:] $a_{\rm G}=4/3$ \quad with $c_0 = \frac{\displaystyle\pi k_{\rm B} T_{\rm c}}{\displaystyle\Delta_0} = \pi/2.14 = 1.468$
\end{description}

\begin{equation}\label{eq:gapT_Manzano}
\Delta(\varphi,T) \simeq \Delta(\varphi)\tanh\left[1.82\left(1.018\left(\frac{T_{\mathrm c}}{T}-1\right)\right)^{0.51}\right]\,.
\end{equation}
The \gapint plug-in calculates $\tilde{I}(T)$ for the following $\Delta(\varphi)$:

\begin{description}
 \item[\textit{s}-wave gap:]
   \begin{equation}
     \Delta(\varphi) = \Delta_0
   \end{equation}
   \musrfit theory line: \verb?userFcn libGapIntegrals TGapSWave 1 2 [3 4]?\\[1.5ex]
    Parameters: $T_{\mathrm c}~(\mathrm{K})$, $\Delta_0~(\mathrm{meV})$, $[c_0~(1),~ a_{\rm G}~(1)]$. If $c_0$ and $a_{\rm G}$ are provided, 
    the temperature dependence according to Eq.(\ref{eq:gapT_Prozorov}) will be used, otherwise Eq.(\ref{eq:gapT_Manzano}) will be utilized.
 \item[\textit{d}-wave gap \cite{Deutscher}:] 
   \begin{equation}
     \Delta(\varphi) = \Delta_0\cos\left(2\varphi\right)
   \end{equation}
   \musrfit theory line: \verb?userFcn libGapIntegrals TGapDWave 1 2 [3 4]?\\[1.5ex]
    Parameters: $T_{\mathrm c}~(\mathrm{K})$, $\Delta_0~(\mathrm{meV})$, $[c_0~(1),~a_{\rm G}~(1)]$. If $c_0$ and $a_{\rm G}$ are provided, 
    the temperature dependence according to Eq.(\ref{eq:gapT_Prozorov}) will be used, otherwise Eq.(\ref{eq:gapT_Manzano}) will be utilized.
 \item[non-monotonic \textit{d}-wave gap \cite{Matsui}:] 
   \begin{equation}
     \Delta(\varphi) = \Delta_0\left[a \cos\left(2\varphi\right) + (1-a)\cos\left(6\varphi\right)\right]
   \end{equation}
    \musrfit theory line: \verb?userFcn libGapIntegrals TGapNonMonDWave1 1 2 3 [4 5]?\\[1.5ex]
    Parameters: $T_{\mathrm c}~(\mathrm{K})$, $\Delta_0~(\mathrm{meV})$, $a~(1)$, $[c_0~(1),~a_{\rm G}~(1)]$. If $c_0$ and $a_{\rm G}$ are provided, 
    the temperature dependence according to Eq.(\ref{eq:gapT_Prozorov}) will be used, otherwise Eq.(\ref{eq:gapT_Manzano}) will be utilized.
 \item[non-monotonic \textit{d}-wave gap \cite{Eremin}:]
    \begin{equation}
      \Delta(\varphi) = \Delta_0\left[\frac{2}{3} \sqrt{\frac{a}{3}}\cos\left(2\varphi\right) / \left( 1 + a\cos^2\left(2\varphi\right)\right)^{\frac{3}{2}}\right],\,a>1/2
    \end{equation}
    \musrfit theory line: \verb?userFcn libGapIntegrals TGapNonMonDWave2 1 2 3 [4 5]?\\[1.5ex]
    Parameters: $T_{\mathrm c}~(\mathrm{K})$, $\Delta_0~(\mathrm{meV})$, $a~(1)$, $a~(1)$, $[c_0~(1),~a_{\rm G}~(1)]$. 
    If $c_0$ and $a_{\rm G}$ are provided, the temperature dependence according to Eq.(\ref{eq:gapT_Prozorov}) will be used, 
    otherwise Eq.(\ref{eq:gapT_Manzano}) will be utilized.
 \item[anisotropic \textit{s}-wave gap \cite{AnisotropicSWave}:] 
    \begin{equation}
      \Delta(\varphi) = \Delta_0\left[1+a\cos\left(4\varphi\right)\right]\,,\,0\leqslant a\leqslant1
    \end{equation}
    \musrfit theory line: \verb?userFcn libGapIntegrals TGapAnSWave 1 2 3 [4 5]?\\[1.5ex]
    Parameters: $T_{\mathrm c}~(\mathrm{K})$, $\Delta_0~(\mathrm{meV})$, $a~(1)$, $[c_0~(1),~a_{\rm G}~(1)]$. 
    If $c_0$ and $a_{\rm G}$ are provided, the temperature dependence according to Eq.(\ref{eq:gapT_Prozorov}) will be used, 
    otherwise Eq.(\ref{eq:gapT_Manzano}) will be utilized.
\item[\textit{p}-wave (point) \cite{Pang2015}:]
   \begin{equation}
     \Delta(\theta, T) = \Delta(T) \sin(\theta) = \Delta(T) \cdot \sqrt{1-z^2}
   \end{equation}
   \musrfit theory line: \verb?userFcn libGapIntegrals TGapPointPWave 1 2 [3 [4 5]]?\\[1.5ex]
    Parameters: $T_{\mathrm c}~(\mathrm{K})$, $\Delta_0~(\mathrm{meV})$, [ \verb?orientation_tag?, $[c_0~(1),~a_{\rm G}~(1)]$]. If $c_0$ and $a_{\rm G}$ are provided, 
    the temperature dependence according to Eq.(\ref{eq:gapT_Prozorov}) will be used, otherwise Eq.(\ref{eq:gapT_Manzano}) will be utilized. \\
    \verb?orientation_tag?: $0=\{aa,bb\}$, $1=cc$, and the default $2=$ average (see Eq.\ (\ref{eq:n_avg}))
\item[\textit{p}-wave (line) \cite{Ozaki1986}:]
   \begin{equation}
     \Delta(\theta, T) = \Delta(T) \cos(\theta) = \Delta(T) \cdot z
   \end{equation}
   \musrfit theory line: \verb?userFcn libGapIntegrals TGapLinePWave 1 2 [3 [4 5]]?\\[1.5ex]
    Parameters: $T_{\mathrm c}~(\mathrm{K})$, $\Delta_0~(\mathrm{meV})$, [ \verb?orientation_tag?, $[c_0~(1),~a_{\rm G}~(1)]$]. If $c_0$ and $a_{\rm G}$ are provided, 
    the temperature dependence according to Eq.(\ref{eq:gapT_Prozorov}) will be used, otherwise Eq.(\ref{eq:gapT_Manzano}) will be utilized. \\
    \verb?orientation_tag?: $0=\{aa,bb\}$, $1=cc$, and the default $2=$ average (see Eq.\ (\ref{eq:n_avg}))
\end{description}

\noindent It is also possible to calculate a power law temperature dependence (in the two fluid approximation $n=4$) and the dirty \textit{s}-wave expression. 
Obviously for this no integration is needed.
\begin{description}
 \item[Power law return function:] 
   \begin{equation}
     \frac{\lambda(0)^2}{\lambda(T)^2} = 1-\left(\frac{T}{T_{\mathrm c}}\right)^n
   \end{equation}
   \musrfit theory line: \verb?userFcn libGapIntegrals TGapPowerLaw 1 2?\\[1.5ex]
   Parameters: $T_{\mathrm c}~(\mathrm{K})$, $n~(1)$
 \item[dirty \textit{s}-wave \cite{Tinkham}:]
    \begin{equation}
      \frac{\lambda(0)^2}{\lambda(T)^2} = \frac{\Delta(T)}{\Delta_0}\,\tanh\left[\frac{\Delta(T)}{2 k_{\rm B} T}\right]
    \end{equation}
    \musrfit theory line: \verb?userFcn libGapIntegrals TGapDirtySWave 1 2 [3 4]?\\[1.5ex]
    Parameters: $T_{\mathrm c}~(\mathrm{K})$, $\Delta_0~(\mathrm{meV})$, $[c_0~(1),~a_{\rm G}~(1)]$. 
    If $c_0$ and $a_{\rm G}$ are provided, the temperature dependence according to Eq.(\ref{eq:gapT_Prozorov}) will be used, 
    otherwise Eq.(\ref{eq:gapT_Manzano}) will be utilized.
\end{description}

\noindent Currently there are two gap functions to be found in the code which are \emph{not} documented here: 
\verb?TGapCosSqDWave? and \verb?TGapSinSqDWave?. For details for these gap functions (superfluid density along the \textit{a}/\textit{b}-axis
within the semi-classical model assuming a cylindrical Fermi surface and a mixed $d_{x^2-y^2} + s$ symmetry of the superconducting order parameter 
(effectively: $d_{x^2-y^2}$ with shifted nodes and \textit{a}-\textit{b}-anisotropy)) see the source code.

\subsection*{Gap Integrals for $\bm{\theta}$-, and $\bm{(\theta, \varphi)}$-dependent Gaps}%

First some general formulae as found in Ref.\,\cite{Prozorov}. It assumes an anisotropic response which can be classified in 3 directions ($a$, $b$, and $c$).

\noindent For the case of a 2D Fermi surface (cylindrical symmetry):

\begin{equation}\label{eq:n_anisotrope_2D}
n_{aa \atop bb}(T) = 1 - \frac{1}{2\pi k_{\rm B} T} \int_0^{2\pi} \mathrm{d}\varphi\, {\cos^2(\varphi) \atop \sin^2(\varphi)} \underbrace{\int_0^\infty \mathrm{d}\varepsilon\, \left\{ \cosh\left[\frac{\sqrt{\varepsilon^2 + \Delta^2}}{2 k_{\rm B}T}\right]\right\}^{-2}}_{= G(\Delta(\varphi), T)}
\end{equation}

\noindent For the case of a 3D Fermi surface:

\begin{eqnarray}
 n_{aa \atop bb}(T) &=& 1 - \frac{3}{4\pi k_{\rm B} T} \int_0^1 \mathrm{d}z\, (1-z^2) \int_0^{2\pi} \mathrm{d}\varphi\, {\cos^2(\varphi) \atop \sin^2(\varphi)} \cdot G(\Delta(z,\varphi), T) \label{eq:n_anisotrope_3D_aabb} \\
 n_{cc}(T) &=& 1 - \frac{3}{2\pi k_{\rm B} T} \int_0^1 \mathrm{d}z\, z^2 \int_0^{2\pi} \mathrm{d}\varphi\, \cos^2(\varphi) \cdot G(\Delta(z,\varphi), T) \label{eq:n_anisotrope_3D_cc}
\end{eqnarray}

\noindent The ``powder averaged'' superfluid density is then defined as

\begin{equation}\label{eq:n_avg}
 n_{\rm S} = \frac{1}{3}\cdot \left[ \sqrt{n_{aa} n_{bb}} + \sqrt{n_{aa} n_{cc}} + \sqrt{n_{bb} n_{cc}} \right]
\end{equation}

\subsubsection*{Isotropic s-Wave Gap}

\noindent For the 2D/3D case this means that $\Delta$ is just a constant. 

\noindent For the 2D case it follows

\begin{equation}
  n_{aa \atop bb}(T) = 1 - \frac{1}{2 k_{\rm B} T} \cdot G(\Delta, T) = n_{\rm S}(T).
\end{equation}

\noindent This is the same as Eq.(\ref{eq:bmw_2d}), assuming a $\Delta \neq f(\varphi)$.

\vspace{5mm}

\noindent The 3D case for $\Delta \neq f(\theta, \varphi)$:

\noindent The variable transformation $z = \cos(\theta)$ leads to $\mathrm{d}z = -\sin(\theta)\,\mathrm{d}\theta$, $z=0 \to \theta=\pi/2$, $z=1 \to \theta=0$, and hence to

\begin{eqnarray*}
 n_{aa \atop bb}(T) &=& 1 - \frac{3}{4\pi k_{\rm B} T} \underbrace{\int_0^{\pi/2} \mathrm{d}\theta \, \sin^3(\theta)}_{= 2/3} \, \underbrace{\int_0^{2\pi} \mathrm{d}\varphi {\cos^2(\varphi) \atop \sin^2(\varphi)}}_{=\pi} \cdot G(\Delta, T) \\
 &=& 1 - \frac{1}{2 k_{\rm B} T} \cdot G(\Delta, T). \\
 n_{cc}(T) &=& 1 - \frac{3}{2\pi k_{\rm B} T} \underbrace{\int_0^{\pi/2} \mathrm{d}\theta \, \cos^2(\theta)\sin(\theta)}_{=1/3} \, \underbrace{\int_0^{2\pi} \mathrm{d}\varphi \cos^2(\varphi)}_{=\pi} \cdot G(\Delta, T) \\
 &=&  1 - \frac{1}{2 k_{\rm B} T} \cdot G(\Delta, T). 
\end{eqnarray*}

\noindent And hence

$$ n_{\rm S}(T) = 1- \frac{1}{2 k_{\rm B} T} \cdot G(\Delta, T). $$

\subsubsection*{3D Fermi Surface Gap $\mathbf{\Delta \neq f(\bm\varphi)}$}

For this case the superfluid density integrals reduce to ($z=\cos(\theta)$)

\begin{eqnarray}
  n_{aa \atop bb}(T) &=& 1 - \frac{3}{4 k_{\rm B} T} \int_0^1 \mathrm{d}z\, (1-z^2) \cdot G(\Delta(z, T),T) \\
  n_{cc}(T) &=& 1 - \frac{3}{2 k_{\rm B} T} \int_0^1 \mathrm{d}z\, z^2 \cdot G(\Delta(z, T),T)
\end{eqnarray}



\subsection*{License}
The \gapint library is free software; you can redistribute it and/or modify it under the terms of the GNU General Public License as published by the Free Software Foundation \cite{GPL}; either version 2 of the License, or (at your option) any later version.

\bibliographystyle{plain}
\begin{thebibliography}{1}

\bibitem{musrfit} A. Suter, and B.M. Wojek, Physics Procedia \textbf{30}, 69 (2012). 
                  A.~Suter, \textit{\musrfit User Manual}, http://lmu.web.psi.ch/musrfit/user/MUSR/WebHome.html
\bibitem{cuba} T.~Hahn, \textit{Cuba -- a library for multidimensional numerical integration}, Comput.~Phys.~Commun.~\textbf{168}~(2005)~78-95, http://www.feynarts.de/cuba/
\bibitem{Prozorov} R.~Prozorov and R.W.~Giannetta, \textit{Magnetic penetration depth in unconventional superconductors}, Supercond.\ Sci.\ Technol.\ \textbf{19}~(2006)~R41-R67, and Erratum in Supercond.\ Sci.\ Technol.\ \textbf{21}~(2008)~082003.
\bibitem{Manzano} A.~Carrington and F.~Manzano, Physica~C~\textbf{385}~(2003)~205
\bibitem{Deutscher} G.~Deutscher, \textit{Andreev-Saint-James reflections: A probe of cuprate superconductors}, Rev.~Mod.~Phys.~\textbf{77}~(2005)~109-135
\bibitem{Matsui} H.~Matsui~\textit{et al.}, \textit{Direct Observation of a Nonmonotonic $d_{x^2-y^2}$-Wave Superconducting Gap in the Electron-Doped High-T$_{\mathrm c}$ Superconductor Pr$_{0.89}$LaCe$_{0.11}$CuO$_4$}, Phys.~Rev.~Lett.~\textbf{95}~(2005)~017003
\bibitem{Eremin} I.~Eremin, E.~Tsoncheva, and A.V.~Chubukov, \textit{Signature of the nonmonotonic $d$-wave gap in electron-doped cuprates}, Phys.~Rev.~B~\textbf{77}~(2008)~024508
\bibitem{AnisotropicSWave} ??
\bibitem{Pang2015} G.M.~Pang, \emph{et al.}, Phys.~Rev.~B~\textbf{91}~(2015)~220502(R), and references in there.
\bibitem{Ozaki1986} M.~Ozaki, \emph{et al.}, Prog.~Theor.~Phys.~\textbf{75}~(1986)~442.
\bibitem{Tinkham} M.~Tinkham, \textit{Introduction to Superconductivity} $2^{\rm nd}$ ed. (Dover Publications, New York, 2004).
\bibitem{GPL} http://www.gnu.org/licenses/old-licenses/gpl-2.0.html

\end{thebibliography}


\end{document}
